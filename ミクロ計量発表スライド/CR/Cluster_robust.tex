\documentclass[xcolor=svgnames,aspectratio=169]{beamer} 
\usetheme{metropolis}
\usefonttheme{professionalfonts}
\setbeamertemplate{theorems}[numbered]
\usepackage{luatexja}
\usepackage{luatexja-fontspec}
\usepackage{newtxtext}                     
\usepackage{amsthm} 
\usepackage{graphicx}
\usepackage{xcolor}
\usepackage{tikz}
\everymath{\displaystyle}
\usepackage{bbm}
\usepackage{lmodern}
\begin{document} 

\title{Cluster-robust inference:\\ \small{A guide to empirical practice}}
\author{Naoki Eguchi}          
\institute{Faculty of Medicine, Kyoto University} 
\date{\today}

\begin{frame}                  
    \titlepage                     
\end{frame}

\section{1. Introduction}

\begin{frame}{English template}
\begin{definition}[name]
Naoki Eguchi
\end{definition}
\end{frame}

\section{2. Cluster-robust variance estimators}

\begin{frame}{2.1. The clustered regression model}
\end{frame}

\begin{frame}{2.2. Three feasible CRVEs}
\end{frame}

\section{3. Why and how to cluster}

\begin{frame}{3.1. Modeling intra-cluster dependence}
\end{frame}

\begin{frame}{3.2. Do cluster fixed effects remove intra-cluster dependence?}
\end{frame}

\begin{frame}{3.3. At what level should we cluster?}
\end{frame}

\begin{frame}{3.4. Leverage and influence}
\end{frame}

\begin{frame}{3.5. Placebo regressions}
\end{frame}

\begin{frame}{3.6. Two-way clustering}
\end{frame}

\section{4. Asymptotic inference}

\begin{frame}{4.1. Asymptotic theory: Large number of clusters}
\end{frame}

\begin{frame}{4.2. Asymptotic theory: small number of large clusters}
\end{frame}

\begin{frame}{4.3. When asymptotic influence can fail}
\end{frame}

\begin{frame}{4.3.1. Cluster heterogeneity}
\end{frame}

\begin{frame}{4.3.2. Treatment and few treated clusters}
\end{frame}

\begin{frame}{4.3.3. Testing several restrictions}
\end{frame}

\begin{frame}{4.4. Cluster-robust inference in nonlinear models}
\end{frame}

\section{5. Bootstrap inference}

\begin{frame}{5.1. General principles of the bootstrap}
\end{frame}

\begin{frame}{5.2. Pairs cluster bootstrap}
\end{frame}

\begin{frame}{5.3. Wild cluster bootstrap}
\end{frame}

\section{6. Other inferential procedures}

\begin{frame}{6.1. Alternative critical values}
\end{frame}

\begin{frame}{6.2. Randomization inference}
\end{frame}

\section{7. What should investigators report?}

\begin{frame}{7. What should investigators report?}
\end{frame}

\section{8. Empirical example}

\begin{frame}{8.1. Cluster heterogeneity}
\end{frame}

\begin{frame}{8.2. Placebo regressions for the empirical example}
\end{frame}

\section{9. Conclusion: A summary guide}

\begin{frame}{9. Conclusion: A summary guide}
\end{frame}

\section{References}

\begin{frame}{References}
    \begin{itemize}
        \item James G. MacKinnon, Morten \UTF{00D8}rregaard Nielsen, Matthew D. Webb, (2023). 
        \textit{Cluster-robust inference: A guide to empirical practice}, Journal of Econometrics.
        \item Alberto Abadie, Susan Athey, Guido W. Imbens, and Jeffrey M. Wooldridge, (2023).
        \textit{When should you adjust standard errors for clustering?}, Quarterly Journal of Economics.
        \item Joahua D. Angrist , J\"{o}rn-steffen Pischke, (2008).
        \textit{Mostly Harmless Econometrics: An Empiricist's Companion}, Princeton University Press.
        \item 川口康平,澤田真行 (2024). 
        \textit{因果推論の計量経済学},日本評論社.
    \end{itemize}
\end{frame}

\end{document}