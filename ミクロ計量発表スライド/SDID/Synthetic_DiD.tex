\documentclass[xcolor=svgnames,aspectratio=169]{beamer} 
\usetheme{metropolis}
\usefonttheme{professionalfonts}
\setbeamertemplate{theorems}[numbered]
\usepackage{amsthm} 
\usepackage{graphicx}
\usepackage{mediabb}
\usepackage{xcolor}
\usepackage{tikz}
\everymath{\displaystyle}
\newtheorem{thm}{Theorem}[section]
\newtheorem{proposition}[thm]{Proposition}
\newtheorem{exam}[thm]{Example}
\newtheorem{remark}[thm]{Remark}
\newtheorem{question}[thm]{Question}
\newtheorem{prob}[thm]{Problem}
\usepackage{bbm}
\usepackage{ascmac}
\usepackage{newtxtext}

%%%%% Document Start %%%%%
\begin{document}

%%%%% Title Slide %%%%%
\title{Synthetic Difference-in-Differences}
\author{Naoki Eguchi}          
\institute{Faculty of Medicine, Kyoto University} 
\date{\today}

\begin{frame}
    \titlepage                     
\end{frame}

%%%%% Content Slides %%%%%
\section{Introduction}

\section{An Application}

\begin{frame}{Implementing SDID}
\end{frame}

\begin{frame}{The California Smoking Cessation Program}
\end{frame}

\section{Placebo Studies}

\begin{frame}{Current Population Survey Placebo Study}
\end{frame}

\begin{frame}{Penn World Table Placebo Study}
\end{frame}

\section{Formal Results}

\begin{frame}{Weighted Double-Differencing Estimators}
\end{frame}

\begin{frame}{Oracle and Adaptive Synthetic Control Weights}
\end{frame}

\section{Large-Sample Inference}

\begin{frame}{Large-Sample Inference}
\end{frame}

\section{Related Work}

\begin{frame}{Related Work}
\end{frame}

\section{References}
\begin{frame}{References}
    \begin{itemize}
        \item Arkhangelsky, D., et al. (2021). 
        \textit{Synthetic Difference-in-Differences}, American Economic Review.
        \item 川口康平, 澤田真行 (2024). 
        \textit{因果推論の計量経済学}, 日本評論社.
\end{itemize}
\end{frame}

\end{document}