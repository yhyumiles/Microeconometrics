\documentclass[xcolor=svgnames,aspectratio=169]{beamer} 
\usetheme{metropolis}
\usefonttheme{professionalfonts}
\setbeamertemplate{theorems}[numbered]
\usepackage{luatexja}
\usepackage{luatexja-fontspec}
\usepackage{newtxtext}                     
\usepackage{amsthm} 
\usepackage{graphicx}
\usepackage{xcolor}
\usepackage{tcolorbox}
\usepackage{tikz}
\everymath{\displaystyle}
\usepackage{bm}
\usepackage{bbm}
\usepackage{lmodern}
\newcommand{\indep}{\mathop{\perp\!\!\!\!\perp}}
\newcommand{\R}{\mathbb{R}} 
\newcommand{\N}{\mathbb{N}} 
\newcommand{\Z}{\mathbb{Z}} 
\newcommand{\Q}{\mathbb{Q}} 
\newcommand{\C}{\mathbb{C}}
\newcommand{\E}{\mathbb{E}}

\begin{document} 

\title{Matrix Completion Methods for Causal Panel Data Models \\ \small{Susan Athey et al. (2021), JASA}}
\author{Naoki Eguchi}          
\institute{Faculty of Medicine, Kyoto University} 
\date{2025.6.25 ミクロ計量経済学}

\begin{frame}                  
    \titlepage                     
\end{frame}

\section{Introduction}

\begin{frame}{What is "matrix completion" ?}
    \begin{itemize}
        \item Matrix completion (MC) is an imputation method for the missing in the matrix.
        \item In PCA (Principle Compoent Analysis), given $M$ principle components, we approximate a true data matrix $\mathbf{Y}\in\mathbb{R}^{I\times J}$ by $\mathbf{A}\in\mathbb{R}^{I\times M}$ and $\mathbf{B}\in\mathbb{R}^{M\times J}$.
        \[
        \min_{\mathbf{A},\,\mathbf{B}}\left\{
        \sum_{i=1}^{I}\sum_{j=1}^{J}\left(Y_{ij}-\sum_{m=1}^{M}a_{im}b_{mj}\right)^2
        \right\}
        \]
        \item If there is missing in matrix, using only observed data, we can approximate the original matrix, hence we can impute the missing by $\mathbf{\hat{A},\hat{B}}$.
        \[
        \min_{\mathbf{A},\,\mathbf{B}}\left\{
        \sum_{(i,j)\in \mathcal{O}}\left(Y_{ij}-\sum_{m=1}^{M}a_{im}b_{mj}\right)^2
        \right\}
        \]
        \item Applying this idea to panel data analysis, MC-based panel method is seen to a generalization of many model-based (factor-regression-based) panel methods.
    \end{itemize}
\end{frame}

\begin{frame}{Where to impute}
    \begin{itemize}
        \item As many panel data methods, we want to know ATT : $\E[Y_{it}(1)-Y_{it}(0)|W_i=1]$.
        \item Thus, it boils down to estimate (impute) the counterfactual $Y_{it}(0)$.
        \begin{itemize}
            \item Horizontal : Under unconfoundedness, we can impute counterfactual PO using observed outcomes for control units.
            \item Vertical : By SCM, we can also impute it using weighted average outcomes for control units with most predictive weights trained with pre-treatment datas.
        \end{itemize}
        \begin{figure}
            \includegraphics[width=0.5\textwidth, height=0.4\textheight, keepaspectratio]{Horizontal.png} \ 
            \includegraphics[width=0.5\textwidth, height=0.5\textheight, keepaspectratio]{Vertical.png}
        \end{figure}
    \end{itemize}
\end{frame}

\begin{frame}{Xu (2024): Counterfactual estimation}
    \begin{itemize}
        \item functional form: 
        $
        Y_{it}(0)=f(\mathbf{X_{it}})+h(\mathbf{U_{it}})+\epsilon_{it}
        $ \\
        \rightarrow No anticipation, carryover, feedback, LDV
        \item strict exogeneity:
        $
        \forall i,j\in\{1,\dots,N\},\ \forall s,t\in\{1,\dots,T\}, \epsilon_{it}\ \indep\ \{D_{js}, \mathbf{X_{js}}, \mathbf{U_{js}}\}
        $
        \begin{figure}
            \includegraphics[width=\textwidth, height=0.4\textheight, keepaspectratio]{Xu_DAG.pdf}
        \end{figure}
        \item low-dimensional decomposition:
        $
        h(\mathbf{U_{it}})=\{L_{it}\}, \mathrm{rank}(\mathbf{L}_{N\times T}) \ll \min\{N,T\}
        $

        →The rank (= number of factors) is \alert{FIXED !!}
        \item MC panel method is a generalization of Xu's counterfactual estimation.
    \end{itemize}
\end{frame}

\section{Set Up}

\begin{frame}{Notation and Estimand}
    \begin{itemize}
        \item Consider a setting with $N$ units observed over $T$ periods characterized by a binary treatment $W_{it}$ and hence two POs $Y_{it}(1), Y_{it}(0)$.
        \begin{itemize}
            \item $\mathbf{X}\in\R^{N\times P} \ , \  \mathbf{Z}\in\R^{T\times Q}$ : observe (unit / time)-specific covariance matrix
        \end{itemize}
        \item Estimand : $\mathbf{Y} = \{Y_{it}(0)\footnote{以降は簡単のため,$Y_{it}(0)=Y_{it}$ とし,“(0)” を省略して表記する.}\}
        =
        \begin{pmatrix}
Y_{11}(0) & \cdots & Y_{1T}(0) \\
\vdots & \ddots & \vdots \\
Y_{N1}(0) & \cdots & Y_{NT}(0)
\end{pmatrix}$ (← Matrix!!)
        \item $W_{it}=
        \begin{cases}
            1 & \text{if} \ (i,t)\in \mathcal{M} : \text{Missing indice} \\ 0 & \text{if} \ (i,t) \in\mathcal{O} : \text{Observed indice as training data}
        \end{cases}$
    \end{itemize}
\end{frame}

\begin{frame}{Patterns of data matrix}
    \begin{itemize}
        \item Ordinary case (rich data wrt. units and times)
        \begin{figure}
            \includegraphics[width=\textwidth, height=0.35\textheight, keepaspectratio]{Ordinary.png}
        \end{figure}
        \item Staggered adoption
        \begin{figure}
            \includegraphics[width=\textwidth, height=0.35\textheight, keepaspectratio]{Staggered_adoption.png}
        \end{figure}
    \end{itemize}
\end{frame}

\begin{frame}{Horizontal regression and unconfoundedness : thin matrix ($N\gg T$)}
    \begin{figure}
            \includegraphics[width=0.5\textwidth, height=0.4\textheight, keepaspectratio]{Horizontal.png} 
        \end{figure}
    \begin{enumerate}
        \item Regress the last period outcome on the lagged outcomes. (among untreated)
        \item Predict the missing POs using the estimated regression.
        \[
        \forall (i,T)\in\mathcal{M} , \ \hat{Y}_{iT}=\hat{\beta}_0+\sum_{t=1}^{T-1}\hat{\beta}_tY_{it}, \ \text{where} \ \ \hat{\beta}=\arg\min_{\beta} \sum_{i:(i,T) \in \mathcal{O} }(Y_{iT}-\beta_0-\sum_{t=1}^{T-1}\beta_tY_{it})^2.
        \]
    \end{enumerate}
    →Nonparametrically, 
\end{frame}

\begin{frame}{Vertical regression and synthesis control : fat matrix ($T\gg N$)}
    \begin{figure}
            \includegraphics[width=0.5\textwidth, height=0.5\textheight, keepaspectratio]{Vertical.png}
    \end{figure}
    \begin{enumerate}
        \item Regress the outcomes for treated unit prior to the treatment on the outcomes for the control units in the same periods.
        \item Predct the missing POs using the estimated regression.
        \[
        \forall (N,t)\in\mathcal{M} , \ \hat{Y}_{Nt}=\hat{\gamma}_0+\sum_{i=1}^{N-1}\hat{\gamma}_iY_{it}, \ \text{where} \ \ \hat{\gamma}=\arg\min_{\gamma} \sum_{t:(N,t) \in \mathcal{O} }(Y_{Nt}-\gamma_0-\sum_{i=1}^{N-1}\gamma_iY_{it})^2.
        \]
    \end{enumerate}
    →Vertical regression is generalization of ADH(2010) in that it relaxes two restrictions :
    \begin{itemize}
        \item the coefficients $\hat{\gamma}$ are nonnegative. (Interpretability ; What is a negative weight?)
        \item the intercept in this regression is $0$. (This is seen to be plausible in recent literatures.)
    \end{itemize}
\end{frame}

\section{Matrix Completion}

\begin{frame}{Model}
    \begin{itemize}
        \item Under no covariates, we model the $N\times T$ matrix of complete matrix $\mathbf{Y}$ as
        \[
        \mathbf{Y}=\mathbf{L^*}+\mathbf{\epsilon}, \ \text{where} \ \ \E[\mathbf{\epsilon}|\mathbf{L^*}]=0.
        \]
    \begin{tcolorbox}[colframe=lightgray,title=Assumption 1]
        \begin{itemize}
            \item $\mathbf{\epsilon}$ is independent of $\mathbf{L^*}$ (strict exogeneity)
            \item The element of $\mathbf{\epsilon}$ are $\sigma-sub-Gaussian$ and independent each other.
            $\Leftrightarrow \forall t, \ \E[\exp (t\epsilon)]\leq \exp (\frac{\sigma^2t^2}{2}).$
        \end{itemize}
    \end{tcolorbox}
        \item The goal is to estimate the matrix $\mathbf{L^*}$. (low-rank assumption) 
    
        →Note that two types\footnote{これら以外にもInteractive fixed effectといったあらゆるfactorを"少数まで"許容する} of fixed effects are included.
    \end{itemize}
\end{frame}

\begin{frame}{MC-NNM (Matrix Completion with Nuclear Norm Minimization) estimator}
    \begin{itemize}
        \item MC-NNM estimator for $\mathbf{L^*}$ is given by $\mathbf{\hat{L}}+\hat{\Gamma}\mathbf{1}_T^{\mathsf{T}} + \mathbf{1}_N\hat{\Delta}^{^\mathsf{T}}$
        \[
        (\mathbf{\hat{L}}, \hat{\Gamma}, \hat{\Delta})=\arg \min_{\mathbf{L},\Gamma,\Delta} 
        \{
            \frac{1}{|\mathcal{O} |}  ||\mathbf{P_{\mathcal{O} }}({\mathbf{Y}-\mathbf{L}-\Gamma}\mathbf{1}_T^{\mathsf{T}}-\mathbf{1}_N\Delta^{\mathsf{T}})||_F^2+\lambda||\mathbf{L}||_*
        \}
        \]
        \begin{itemize}
            \item $\Gamma\in\R^N$ : unit-varying (and time-fixed) effect (individual effect)
            \item $\Delta\in\R^T$ : time-varying (and unit-fixed) effect (time effect) 
            \item matrix indicator function : $\mathbf{P_{\mathcal{O} }}(\mathbf{A})=
            \begin{cases}
                A_{it} & \text{if} \ (i,t)\in\mathcal{O} \\ 0 & \text{if} \ (i,t) \notin \mathcal{O}  
            \end{cases}$ \ (NA is regarded as $0$)
            \item Frobenius norm : $||\mathbf{A}||_F^2=\sum_{i=1}^N\sum_{t=1}^T A_{it}^2$ \ (行列版のmean squared errorを計算している)
        \end{itemize}
        \item Regularization term $\lambda||L||_*$ leads to \alert{the low rank of $\mathbf{L}$}. \\
        → minimize $\lambda||L||_*$ $\Leftrightarrow $ the sparsity of Singular value $\sigma_i(\mathbf{L})(>0)$ $\Leftrightarrow $ low rank of $\mathbf{L}$
    \end{itemize}
\end{frame}

\begin{frame}
    \begin{itemize}
        \item \textbf{Fact 1.} (Singular value decomposition) \textit{Every real matrix $L\in\R^N\times \R^T$ can be decomposed using a onthogonal matrix $\mathbf{S}\in \R^N\times \R^{\min(N,T)},\mathbf{R}\in\R^T\times \R^{\min(N,T)}$ by}
        \[
        \mathbf{L}=\mathbf{S}\mathbf{\Sigma}\mathbf{R'}, \text{where} \ \mathbf{\Sigma}=\text{diag}(\sigma_1,\dots\sigma_{\min(N,T)}), \mathbf{S'S=I_{\min(N,T)}=R'R}
        \]
        \item \textbf{Fact 2.} \textit{The number of non-zero singular value = rank $\mathbf{L}$}
        \begin{itemize}
            \item Nuclear norm : $||L||_*=\sum_{i=1}^{\min(N,T)} \sigma_i(\mathbf{L})$ \\
            → minimize $\lambda||L||_*$ $\Leftrightarrow $ the sparsity of Singular value $\sigma_i(\mathbf{L})(>0)$ $\Leftrightarrow $ low rank of $\mathbf{L}$
        \end{itemize}
        \item Since the rank of $\mathbf{L}$ corresponds to \alert{the number of factor}, this assumption of low rank is quite plausible. 
        \item Although the law rank matrix CAN include two fixed effects, these "strong" factors are separately estimated for improving the quality of the practical imputations.
    \end{itemize}
\end{frame}

\begin{frame}{Algorithm for calculating $\hat{\mathbf{L}}$}
    \begin{itemize}
        \item For simplicity, assume that there are no fixed effects. (only estimate $\mathbf{L}$)
        \item \textbf{Fact 3.} \textit{For $\mathbf{A=S\Sigma R}$, the minimizer is obtained analytically.}
        \[
        \mathbf{S\tilde{\Sigma}R^\mathsf{T}}=\arg\min_{\mathbf{A}}\{\frac{1}{2}||\mathbf{L-A}||_F^2+\lambda||\mathbf{A}||_*\}, \text{where} \ \tilde{\Sigma}=\text{diag}(\{\max(\sigma_i(\mathbf{A})-\lambda,0)\}_i)
        \]
        → You can see elements with a small singular value (=weak factor) will be vanished.
        \item We perform this minimization over and over until the matrix converges.
        \begin{itemize}
            \item Define $\text{shrink}_{\lambda}(\mathbf{A})=\mathbf{S\tilde{\Sigma}R^\mathsf{T}}$ and start with the initial choice $\mathbf{L_1}(\lambda, \mathcal{O})=\mathbf{P}_\mathcal{O} (\mathbf{Y})$ (The missing starts with $0$.)
            \[
            \mathbf{L}_{k+1}(\lambda,\mathcal{O} )=\text{shrink}_{\frac{\lambda|\mathcal{O} |}{2}}\{\mathbf{P}_{\mathcal{O} }(\mathbf{Y})+\mathbf{P}_{\mathcal{O} }^\mathsf{T}(\mathbf{L}_k(\lambda,\mathcal{O} ))\}
            \]
            \item $\mathbf{P_{\mathcal{O} }}(\mathbf{A})=
            \begin{cases}
                A_{it} & \text{if} \ (i,t)\in\mathcal{O} \\ 0 & \text{if} \ (i,t) \notin \mathcal{O}  
            \end{cases}$ \ , \quad $\mathbf{P_{\mathcal{O} }^\mathsf{T}}(\mathbf{A})=
            \begin{cases}
                0 & \text{if} \ (i,t)\in\mathcal{O} \\ A_{it} & \text{if} \ (i,t) \notin \mathcal{O}  
            \end{cases}$
            \item The limiting matrix $\hat{\mathbf{L}}(\lambda,\mathcal{O} )=\lim_{k\to \infty} \mathbf{L}_k(\lambda,\mathcal{O} )$ is MC-NNM estimator given $\lambda$.
        \end{itemize}
    \end{itemize}
\end{frame}

\begin{frame}
    \begin{itemize}
        \item For the case with fixed effects, we replace $\mathbf{P}_{\mathcal{O} }(\mathbf{Y})$ with $\mathbf{P_{\mathcal{O} }}({\mathbf{Y}-\Gamma_k}\mathbf{1}_T^{\mathsf{T}}-\mathbf{1}_N\Delta_k^{\mathsf{T}})$
        \item After each iteration to obtain $\hat{\mathbf{L}}_{k+1}$, we can estimate $\Gamma_{k+1}$ and $\Delta_{k+1}$ by using the first-order conditions.
        \begin{itemize}
            \item More specifically, after estimating $\hat{\mathbf{L}}_{k+1}$, the objective function is as the quadratic form wrt. $\Gamma_k, \Delta_k$, so we additionally minimize it and renew $\Gamma_{k+1}, \Delta_{k+1}$.
            \item Finally, replace the $\mathbf{P_{\mathcal{O} }}({\mathbf{Y}-\Gamma_k}\mathbf{1}_T^{\mathsf{T}}-\mathbf{1}_N\Delta_k^{\mathsf{T}})$ with $\mathbf{P_{\mathcal{O} }}({\mathbf{Y}-\Gamma_{k+1}}\mathbf{1}_T^{\mathsf{T}}-\mathbf{1}_N\Delta_{k+1}^{\mathsf{T}})$, \\ then proceed the algorithm to obtain $\hat{\mathbf{L}}_{k+2}$.
        \end{itemize}
        \item we can interpert the term $\mathbf{P_{\mathcal{O} }}({\mathbf{Y}-\Gamma_k}\mathbf{1}_T^{\mathsf{T}}-\mathbf{1}_N\Delta_k^{\mathsf{T}})$ as an \alert{invariant (fixed)} factor term.
        \item Actually, this separation of two FEs makes practical imputation greatly improved.
    \end{itemize}
\end{frame}

\begin{frame}{Inference : CV and CI}
    \begin{itemize}
        \item The optimal value of $\lambda$ is selected through \alert{cross-validation}.
        \item The theory of asymptotic distribution of $\mathbf{L^*}-\mathbf{\hat{L}}$ to construct CI has not yet developed.
        \item Instead, using resampling method, we see the fluctulation of the imputed matrix and construct CI like premutation methods in ADH-synthetic control method.
        \begin{itemize}
            \item Randomly select the subset $\mathcal{O}_k$ of $\mathcal{O} $ for $k=1,...,K$.
            \item Then, using the sample in $ \mathcal{O}_k$, calculate MC-NNM estimator $\hat{\mathbf{L}}^{(k)}$.
            \item Finally, we construct a pointwise CI for $\{\hat{L}_{it}^{(k)}\}$.
        \end{itemize}
        \item However, according to Choi and Yuan(2024) which is the extended literature of Athey et al.(2021), their \alert{debiased} estimator for $\mathbf{L}^*$ is \alert{asymptotic normal} (but pointwisely).
    \end{itemize}
\end{frame}

\section{The relationship with horizontal and vertical regressions}

\begin{frame}{Interpretation as generalization}
    \begin{itemize}
        \item For simplicity, we assume when only one missing $Y_{NT}(0)$ exists i.e. $\mathcal{M}=(N,T)$ 
        \item Let the estimand $\mathbf{Y}$ partition as $
        \begin{pmatrix} \mathbf{Y_0} & \mathbf{y_1} \\ \mathbf{y_2}^{\mathsf{T}} & ? \end{pmatrix}$, \\ where $\mathbf{Y_0}\in\R^{N-1}\times \R^{T-1}, \mathbf{y_1}\in\R^{N-1}, \mathbf{y_2}\in\R^{T-1}$.
        \item For a given positive integer $R$, define an $N\times R$ matrix $\mathbf{A}$, an $T\times R$ matrix $\mathbf{B}$, a $N$-dim. vector $\gamma$ and a $R$-dim. vector $\delta$, then, the objective function w.r.t. MSE is
        \[
        Q(\mathbf{Y}; R, \mathbf{A}, \mathbf{B}, \gamma, \delta)=\frac{1}{|\mathcal{O} |} ||\mathbf{P_{\mathcal{O} }}({\mathbb{Y}-\mathbf{AB}^{\mathsf{T}}-\gamma}\mathbf{1}_T^{\mathsf{T}}-\mathbf{1}_N\delta^{\mathsf{T}})||_F^2
        \]
        \item However, there is no unique solution for $\mathbf{A, B}$ unless restrict them. \\
        → Thus, we compare the \alert{restriction} of each methods.
    \end{itemize}
\end{frame}

\begin{frame}{MC-NNM estimator}
    \begin{itemize}
        \item Nuclear norm matrix completion
        \begin{align*}
        (R_\lambda^{\mathrm{mc\text{-}nnm}},\,
         \mathbf{A}_\lambda^{\mathrm{mc\text{-}nnm}},\,
         \mathbf{B}_\lambda^{\mathrm{mc\text{-}nnm}},\,
         \gamma_\lambda^{\mathrm{mc\text{-}nnm}},\,
         \delta_\lambda^{\mathrm{mc\text{-}nnm}})& \\
        = \arg\min_{R,A,B,\gamma,\delta}
           \Bigl\{Q(\mathbf{Y};R,\mathbf{A},\mathbf{B},\gamma,\delta)
        &+\frac{\lambda}{2}\|\mathbf{A}\|_F^2
           +\frac{\lambda}{2}\|\mathbf{B}\|_F^2\Bigr\}.
        \end{align*}
        \item \textbf{Fact 3.} $\|\mathbf{L}\|_*
        = \min_{\mathbf{A},\mathbf{B}:\,\mathbf{L}=\mathbf{A}\mathbf{B}'}
          \frac{1}{2}\bigl(\|\mathbf{A}\|_F^2 + \|\mathbf{B}\|_F^2\bigr).$
        \item In second and third terms, we regularize $\mathbf{A, B}$ so that the minimization problem has a unique solution. \\
        → Compared with other methods, there is a big differnece in that MC-NNM does not restrict the form of the matrix but just regularize it in a data-driven manner.
  \end{itemize}
\end{frame}

\begin{frame}{Horizontal regression (thin matrix, $N>T$)}
  \begin{itemize}
    \item Horizontal regression estimator, defined if $N>T$ : thin matrix
      \begin{align*}
        &(R^{hr},\,\mathbf{A}^{hr},\,\mathbf{B}^{hr},\,\gamma^{hr},\,\delta^{hr})
        = \arg\min_{R,A,B,\gamma,\delta}
           Q\bigl(\mathbf{Y};R,\mathbf{A},\mathbf{B},\gamma,\delta\bigr),\\
        &\text{subject to}\quad
        R = T-1,\quad
        \mathbf{A} = \begin{pmatrix}\mathbf{Y}_0 \\ \mathbf{y}_2^\top\end{pmatrix},\quad
        \gamma = 0,\quad
        \delta_1 = \cdots = \delta_{T-1} = 0.
      \end{align*}
    \item The solution for $\mathbf{B}$ is
    \[
      \mathbf{B}^{hr\top}
      = \begin{pmatrix} E_{T-1} & \hat\beta \end{pmatrix},
      \quad
      (\hat\beta,\hat\delta_T)
      = \arg\min_{\beta,\delta_T}
        \sum_{i=1}^{N-1}
        \bigl(Y_{iT} - \delta_T - \sum_{t=1}^{T-1}\beta_t\,Y_{it}\bigr)^2.
    \]
    \item Restrict the form of $\mathbf{A}$ and (of course) assume no individual effect ($\gamma=0$).
  \end{itemize}
\end{frame}


\begin{frame}{Vertical regression (fat matrix, $T>N$)}
    \begin{itemize}
        \item Vertical regression estimator, defined if $T>N$ : fat matrix
      \begin{align*}
        &(R^{vt},\,\mathbf{A}^{vt},\,\mathbf{B}^{vt},\,\gamma^{vt},\,\delta^{vt})
        = \arg\min_{R,A,B,\gamma,\delta}
           Q\bigl(\mathbf{Y};R,\mathbf{A},\mathbf{B},\gamma,\delta\bigr),\\
        &\text{subject to}\quad
        R = N-1,\quad
        \mathbf{B}^{\mathsf{T}} = \begin{pmatrix}\mathbf{Y}_0 & \mathbf{y}_1\end{pmatrix},\quad
        \gamma_1 = \cdots = \gamma_{N-1} =0 ,\quad
        \delta= 0.
      \end{align*}
      \item The solution for $\mathbf{A}$ is
    \[
      \mathbf{A}^{\text{vt}}
      = \begin{pmatrix} E_{N-1} \\ \hat\alpha \end{pmatrix},
      \quad
      (\hat\alpha,\hat\gamma_N)
      = \arg\min_{\alpha,\gamma_N}
        \sum_{t=1}^{T-1}
        \bigl(Y_{Nt} - \gamma_N - \sum_{i=1}^{N-1}\alpha_i\,Y_{it}\bigr)^2.
    \]
    \item Restrict the form of $\mathbf{B}$ and (of course) assume no time effect ($\delta=0$).
    \end{itemize}
\end{frame}

\begin{frame}{Another methods}
    \begin{figure}[h]
  \centering
  \begin{minipage}{0.43\columnwidth}
    \centering
    \includegraphics[width=0.95\textwidth, height=0.95\textheight, keepaspectratio]{SC.png}
  \end{minipage}
  \hspace{5mm}
  \begin{minipage}{0.43\columnwidth}
    \centering
    \includegraphics[width=0.95\textwidth, height=0.95\textheight, keepaspectratio]{Vt_EN.png}
  \end{minipage}
  \end{figure}
\end{frame}

\section{Theoritical Bounds \\ for the Estimation Error}

\begin{frame}{Estimation error}
    \begin{itemize}
        \item Define $p_c$ to be a minimiun expected proportion of control (never-treated) unit.
        \[
        p_c\equiv \min_{1\leq i\leq N}\pi_T^{(i)}, \text{where} \ \pi_T^{(i)}=\mathbb{P}(t_i=T)
        \]
        \begin{tcolorbox}[colframe=lightgray,title=Assumption 2]
        \begin{itemize}
            \item $p_c$ is sufficiently large even as $N,T \to \infty$ with $N\geq T$. (remained control units)
            \[
            p_c \gtrsim \frac{\log^{3/2}(N+T)}{\sqrt{T}} \vee \frac{\sqrt{T}\log^{3/2}(N+T)}{N}
            \]
            \item True matrix $\mathbf{{L}}^*$ is a low rank.
        \end{itemize}
        \end{tcolorbox}
        \item Under this assumption, $\frac{||\mathbf{{L}^*-\mathbf{\hat{L}}}||_F}{\sqrt{NT}}$ has an upper bound with a high probability ($\mathbb{P}=1-\frac{2}{(N+T)^2}$), then, MC-NNM estimator has a \alert{consisitency}.
    \end{itemize}
\end{frame}

\section{Empirical application}

\begin{frame}{Comparing each method}
    \begin{itemize}
        \item In a real data matrix $\mathbf{Y}$ where \alert{no units is treated}, we choose units as "hypothetical treated" units (=regard as missing) and aim to predict (impute) their value. \\
        → Technically, compare the average root-mean-squared-error(RMSE) to assess which of the algorithm generally perform well.
        \item The compared estimators (algorithms) are as follows.
        \begin{itemize}
            \item DiD : 2回差分によりcounterfactualを補完する
            \item VT-EN : elastics netでvertical regression functionを作り、その推定値で補完
            \item HR-EN : elastics netでhorizontal regression functionを作り、その推定値で補完
            \item SC-ADH : classical approach to construct "synthetic control".
            \item MC-NNM : 本日の主役
        \end{itemize}
    \end{itemize}
\end{frame}

\begin{frame}{ADH(2010): California smoking data}
    \begin{itemize}
        \item $N=38, T=31$
        \begin{itemize}
            \item Simultaneous adoption : Let randomly selected $N_t=8$ units "treated" in period $T_0+1$.
            \item Staggered adoption : Let randomly selected $N_t=35$ units "treated" in some periods after period $T$.
            \begin{figure}
            \includegraphics[width=0.7\textwidth, height=0.4\textheight, keepaspectratio]{ADH.png}
            \end{figure}
        \end{itemize}
        \item (a) : VT-EN performs well on the whole, DiD is poor.
        \item (b) : MC-NNM is superior (large sample is favorable!), and VT-EN is generally poor.
    \end{itemize}
\end{frame}

\section{Final marks}

\begin{frame}{Summary and Conclusion}
    \begin{itemize}
        \item Matrix completion is a imputing method for the missing $Y_{it}(0)$ where $(i,t)\in \mathcal{M}$.
        \item MC-NNM estimator generalizes and nests many estimators such as DiD, ADH-SC, vertical or horizontal (penalized) regression, and interactive fixed effect model.
        \item The critical differnce with previous DiD estimators is that MC-NNM \alert{holistically regularizes} latent factors through nuclear-norm minimization which induces a low-rank matrix (sparsity).
        \begin{itemize}
            \item Intrinsic factor vs. Explicitly imposed factor
        \end{itemize}
        \item Under appropriate conditions (low-rank structure, sufficient number of control units, sufficiently large $N,T$), the MC-NNM estimator achieves \alert{consistency}. 
        \item Practically, this estimator performs well with large $N$ and $T$, and allows for a relatively large number of factors.
    \end{itemize}
\end{frame}

\begin{frame}{Further extensions}
    \begin{itemize}
        \item To consider the model with covariates, we can separately include them in $\frac{1}{|\mathcal{O} |} ||\mathbf{P_{\mathcal{O} }}({\mathbf{Y}-\mathbf{AB}^{\mathsf{T}}-\gamma}\mathbf{1}_T^{\mathsf{T}}-\mathbf{1}_N\delta^{\mathsf{T}})||_F^2$.
        \item For further advanced developments, a recent study by Choi and Yuan (2024, JASA) proposes a debiased estimator for $\mathbf{L^*}$ with rigorous asymptotic inference (pointwise asymptotic normality). 
        \item They introduce a \alert{cross-fitting} (sample-splitting) strategy and further propose group-based ATT estimators, extending the matrix completion literature toward causal inference with stronger theoretical guarantees.
    \end{itemize}
\end{frame}

\section{References}

\begin{frame}{References}
    \begin{itemize}
        \item Susan Athey, Mohsen Bayati, Nikolay Doudchenko, Guido Imbens, and Khashayar Khosravi (2021), \textit{Matrix Completion Methods for Causal Panel Data Models}, Journal of the American Statistical Association.
        \item Licheng Liu, Ye Wang, and Yiqing Xu (2024), \textit{A Practical Guide to Counterfactual Estimators for Causal Inference with Time-Series Cross-Sectional Data},  American Journal of Political Science.
        \item Jungjun Choi and Ming Yuan(2024), \textit{Matrix Completion When Missing Is Not at Random and Its Applications in Causal Panel Data Models}, Journal of the American Statistical Association.
        \item Martin J. Wainwright (2019), \textit{High-Dimensional Statistics}, Cambridge University Press. (mainly Chapter 10)
        \item 冨岡亮太 (2015), スパース性に基づく機械学習, 講談社.
        \item 川口康平,澤田真行 (2024), 因果推論の計量経済学,日本評論社.
    \end{itemize}
\end{frame}

\end{document}